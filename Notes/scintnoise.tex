\documentclass{article}

\begin{document}

pulsar intensities are $\chi^2$ distributed with two degrees of
freedom (Gwinn XXX).  
\begin{equation}
P(I)=\left\{\begin{array}{ll}
\exp(-I/I_0)/I_0 \ \ \ \ & I>0 \\
             0 & {\rm otherwise}
\end{array}
\right.
\end{equation}
Our goal is to measure the mean flux $I_0$ (or any other S/N dependent
quantity such as ToA).  A linear estimator from $n$ samples is
\begin{equation}
\langle I_0 \rangle = \frac{\sum_{i} I_i}{n}
\end{equation}
We model each sample to have a measurement noise $\sigma$, and that
the pulsar does not dominate $T_{\rm sys}$.  The optimal weight is
$I$, giving mean and variances
\begin{equation}
\langle \Delta I_0^2 \rangle = \frac{\sum_{i=0}^{n-1} I^2_i}{n}
\end{equation}


\end{document}